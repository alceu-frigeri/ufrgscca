%%%==============================================================================
%% Copyright 2022-23 by Alceu Frigeri
%%
%% This work may be distributed and/or modified under the conditions of
%%
%% * The [LaTeX Project Public License](http://www.latex-project.org/lppl.txt),
%%   version 1.3c (or later), and/or
%% * The [GNU Affero General Public License](https://www.gnu.org/licenses/agpl-3.0.html),
%%   version 3 (or later)
%%
%% This work has the LPPL maintenance status *maintained*.
%%
%% The Current Maintainer of this work is Alceu Frigeri
%%
%% This is version 1.11 (2023/05/29)
%%
%% The list of files that compose this work can be found in the README.md file at
%% https://ctan.org/pkg/ufrgscca
%%
%%%==============================================================================
%% UFRGS stands for "Federal University of Rio Grande do Sul" in south Brazil
%% EE    stands for "Engineering School"
%% CCA   stands for "Control and Automation Engineering Course" (Portuguese acronym)
%%%==============================================================================
\makeatletter
\ExplSyntaxOff

%%%%%%%%%%%%%%%
%%%%%%%%%%%%%%%

\SetHeadings{%
    university              = Universidade Federal do Rio Grande do Sul ,
    acronym                 = UFRGS ,
    unit                    = Escola de Engenharia ,
    secretary               = SACAD/EE ,
    course                  = Engenharia de Controle e Automação ,
}

\SetTermKeys{%
  title.tccI                = Trabalho de Conclusão de Curso I ,
  title.tccII               = Trabalho de Conclusão de Curso II ,  
  title.internship          = Estágio Supervisionado Obrigatório ,
  %
  title.requestform         = Formulário de Requerimento de Matrícula ,
  title.coadvisorform       = Requerimento de Autorização de Co-Orientação ,
  title.boardapproval       = Formulário de Aprovação da Banca Examinadora ,
  title.advisorsapproval-I  = Formulário de Aprovação para Avaliação ,
  title.advisorsapproval-II = Formulário de Aprovação para Apresentação ,
  title.receipts            = Recibo de entrega de cópia da versão final do Trabalho de Conclusão ,
  title.reviewersform       = Formulário de Avaliação ,
  title.correctionsform     = Formulário de Correções ,
  title.rectifyapproval     = Formulário de Aprovação das Correções ,
  title.internsupervisorform = Formulário de Avaliação do Supervisor (Pela Empresa) ,
  title.interntutorform      = Formulário de Avaliação do Tutor e Coordenador ,
  %
  %footnote.reviewersform
  footnote.reviewersform    = {Entregar a cada membro da banca juntamente com a cópia do trabalho.} , %%% TCC I / II
  footnote.receipts         = {Entregar a cada membro da banca juntamente com a cópia do trabalho final e após encaminhar à coordenação de TCC.} , %%% TCC II
  footnote.correctionsform  = {Entregar a cada membro da banca juntamente com a cópia do trabalho. Após apresentação, retornar este à coordenação de TCC.} ,%%% TCC II
   %
  footnote.supervisorsform = {Favor utilizar a seguinte grade discreta: \begin{enumerate*}[start=0]\item Sofrível \item Regular \item Bom \item Muito Bom \item Excelente \end{enumerate*}} ,
   %
  opening.requestform       = { Ilm\ActivityCoord{course}{article}. Sr\ActivityCoord{course}{narticle}.\par
 Coordenador\ActivityCoord{course}{narticle} da  Comissão de Graduação do Curso em \Activity{course}{name}\  da \UseHeading{university}.   } ,
  localdate                 = {\tc@city,\hspace{4mm} de \hspace{1.2cm} de 20\hspace{4mm}.} ,
  %
  %
  %
  %
 footnote.requestform       = { O resumo deve ser curto, i.e., não deve exceder este espaço. O texto deve ser escrito em parágrafo único, 
                                e por se tratar de proposta, poderá utilizar  tempo futuro e não necessita ter motivação ou justificativa.\break  Deve conter:
                                \begin{enumerate*}[alpha)] 
                                	\item O QUE será feito - descrição e objetivos.
                                   \item COMO será feito - metodologia.
                                	\item Resultados esperados.
                                \end{enumerate*}%
                                } ,   %%% TCC I
  desc.requestform          = { \DataFields{student}{Article} alun\DataFields{student}{article} \DataFields{student}{name}, 
                              REGULARMENTE MATRICULAD\DataFields{student}{Article} neste período letivo no curso \emph{\Activity{course}{name}}, desta universidade, 
                              portador\DataFields{student}{narticle} do cartão de identificação n$^o$ \DataFields{student}{ID}, 
                              vem por meio deste solicitar matrícula na atividade de ensino} ,
  desc.coadvisorform            = {\DataFields{student}{Article} alun\DataFields{student}{article} \DataFields{student}{name}, 
                              REGULARMENTE MATRICULAD\DataFields{student}{Article} na atividade de TCC neste período letivo, 
                              vem por meio deste solicitar autorização para registro de co-orientação de TCC, conforme justificativa abaixo.} ,
  desc.rectifyapproval      = { \begin{description}
                              \item[Observação 1:] Caso a versão final corrigida do trabalho, mais este formulário, não sejam entregues até a data estipulada, o conceito final passará a ser {\bf D}.
                              \item[Observação 2:] Caso as correções sejam satisfatórias, a nota final será a já atribuída pela Banca Examinadora. 
                              \item[Observação 3:] A assinatura d\DataFields{student.advisor}{article} orientador\DataFields{student.advisor}{narticle} é obrigatória, entretanto a assinatura dos membros da banca só é necessária caso tenham marcado a opção ''$\Box$ Revisarei o trabalho depois de corrigido'' no \emph{\UseTerm{temp.formrefname}}.
                            \end{description}
                              } ,
  desc.studentwork          = {
                              \noindent\UseLabel{student}: {\bf \csname tc@student\Alph{tc@studentcount}\endcsname}\hfill \UseLabel{student.ID}: {\bf \DataFields{student}{ID}}\\%
                              \noindent\UseLabel{worktitle}: \@ifundefined{tc@student\Alph{tc@studentcount}title}{\textsl{\@title}}{\textsl{\csname tc@student\Alph{tc@studentcount}title\endcsname}}} ,
  desc.studentwork.full     = {\UseTerm{desc.studentwork}\\[2.5mm]\UseTerm{desc.advisor}} ,  
  desc.advisor              = {\noindent\UseLabel{advisor}:  \textbf{\csname tc@student\Alph{tc@studentcount}@advisor\endcsname}} ,
  desc.coadvisor            = {\noindent\UseLabel{coadvisor}:  \textbf{\csname tc@student\Alph{tc@studentcount}@advisor\endcsname}} ,
  desc.reviewer             = {\noindent\UseLabel{reviewer}:  \textbf{\csname tc@student\Alph{tc@studentcount}@advisor\endcsname}} ,
  desc.student              = {\noindent\UseLabel{student}: {\bf \DataFields{student}{name}}\hfill 
                              \UseLabel{student.ID}: {\bf \DataFields{student}{ID}}} ,
  desc.internshipperiod     = {\noindent Período do Estágio: {\bf \DataFields{student.internship}{start}} até: {\bf \DataFields{student.internship}{end}}} ,
  desc.supervisor           = {\UseLabel{supervisor}: \textbf{\DataFields{student.supervisor}{name}} \hfill CREA: \textbf{\DataFields{student.supervisor}{register}}} ,
  corrections.ok            = {Correções satisfatórias: \hspace*{1cm}$\Box$\ sim\hspace*{2cm}$\Box$\ não} , 
}

\SetTermKeys{%
  notes.calendar            = {%
    \begin{description}
      \item[Obs. 1]: Caso o aluno(a) já tenha um processo aberto de TCC, reusar o mesmo apenas anexando o novo requerimento de matrícula, caso necessário.
      \item[Obs. 2]: O formulário deve conter as assinaturas de concordância em participar da banca da parte dos professores convidados.
      \item[Obs. 3]: As bancas serão marcadas após a entrega do relatório parcial, de acordo com a disponibilidade dos professores convidados.
      \item[Obs. 4]: O \emph{Formulário de Aprovação das Correções} deve estar assinado pelo(a) orientador(a). Para relatórios em que o campo “Revisarei o trabalho depois de corrigido” tiver sido assinalado por algum membro da banca, a assinatura do(a) mesmo(a) também deverá constar no \emph{Formulário de Aprovação das Correções}.
      \item[Obs. 5]: Trabalho em \textbf{RECUPERAÇÃO} é aquele cuja nota final quando da defesa é insuficiente para aprovação OU aquele em que o campo “Necessita ser revisto/re-escrito” do formulário de correções foi marcado por algum membro da banca. Neste caso, o \emph{Formulário de Aprovação das Correções} deve estar assinado tanto pelo(a) orientador(a) como pelo(s) membro(s) da banca designado(s) para reavaliar o trabalho, incluindo obrigatoriamente aqueles que tiverem marcado o campo “Revisarei o trabalho depois de corrigido”.\\\hrule
      \item[IMPORTANTE]: O(A) aluno(a) estará \textbf{REPROVADO(A)} \emph{se falhar na entrega de qualquer um dos documentos aqui citados dentro dos prazos estipulados}.
    \end{description}
  } ,
}

\SetLabels{%
  advisor                   = Orientador\DataFields{student.advisor}{narticle} ,
  advisor.gen               = Orientador(a) ,
  advisor.email             = Email d\DataFields{student.advisor}{article} Orientador\DataFields{student.advisor}{narticle} ,
  coadvisor                 = Co-orientador\DataFields{student.coadvisor}{narticle} ,
  coadvisor.gen             = Co-orientador(a) ,
  coadvisor.email           = Email d\DataFields{student.coadvisor}{article} Co-orientador\DataFields{student.coadvisor}{narticle} ,
  reviewer.base             = Avaliador ,
  reviewer                  = Avaliador\DataFields{student.reviewer}{narticle} ,
  altreviewer               = Suplente ,  
  altreviewer.sig           = Avaliador\DataFields{student.altreviewer}{narticle} ,
  tutor                     = Tutor\DataFields{student.tutor}{narticle} ,
  tutor.gen                 = Tutor(a) ,
  tutor.email               = Email d\DataFields{student.tutor}{article} Tutor\DataFields{student.tutor}{narticle} ,  
  supervisor                = Supervisor\DataFields{student.supervisor}{narticle} ,
  supervisor.gen            = Supervisor(a) ,  
  student                   = Alun\DataFields{student}{article} ,
  student.gen               = Aluno(a) ,
  student.email             = Email d\DataFields{student}{article} Alun\DataFields{student}{article} ,
  student.ID                = Cartão ,
  signature.gen             = Assinatura , 
  signature.student         = Assinatura\ \ d\DataFields{student}{article}\ \ Alun\DataFields{student}{article} ,
  signature.student.gen     = Assinatura\ \ do(a)\ \ Aluno(a) ,
  signature.advisor         = Assinatura\ \ d\DataFields{student.advisor}{article}\ \ Orientador\DataFields{student.advisor}{narticle} ,
  signature.coadvisor       = Assinatura\ \ d\DataFields{student.coadvisor}{article}\ \ Co-orientador\DataFields{student.coadvisor}{narticle} ,
  signature.reviewer        = Assinatura\ \ d\DataFields{student.reviewer}{article}\ \ Avaliador\DataFields{student.reviewer}{narticle} ,
  signature.board           = Assinatura\ \ do\ \ Membro\ \ da\ \ Banca ,
  signature.gen             = Assinatura ,
  signature.supervisor      = Assinatura\ \ d\DataFields{student.supervisor}{article}\ \ Supervisor\DataFields{student.supervisor}{narticle} ,
  signature.tutor           = Assinatura\ \ d\DataFields{student.tutor}{article}\ \ Tutor\DataFields{student.tutor}{narticle} ,
  signature.internshipcoord = Assinatura\ \ d\ActivityCoord{internship}{article}\ \ Coordenador\ActivityCoord{internship}{narticle} ,
  worktitle                 = Título do Trabalho ,
  worktitle.temp            = Título Provisório ,
  email                     = Email ,
  phone                     = ramal ,
  board                     = Banca Examinadora ,
  board.member              = Membro da Banca ,
  agree                     = De Acordo ,
}

\SetLabels{%
  TCC-I.pointA              = {Tema} ,
  TCC-I.pointB              = {Biblio.} ,
  TCC-I.pointC              = {Plano Trab.} ,
  TCC-I.pointD              = {Texto} ,
  TCC-I.weightA              = {2} ,
  TCC-I.weightB              = {2} ,
  TCC-I.weightC              = {2} ,
  TCC-I.weightD              = {1} ,
  TCC-II.pointA              = {Conteúdo} , 
  TCC-II.pointB              = {Trab. Escrito} ,
  TCC-II.pointC              = {Apresentação} ,
  TCC-II.weightA              = {0.35} , 
  TCC-II.weightB              = {0.35} ,
  TCC-II.weightC              = {0.3} 
}


\SetTermKeys{%
}




%%%%%%%%%%%%%%%%%%%%%%%%%%%%%%%%%%%%
%%%%%%%%%%%%%%%%%%%%%%%%%%%%%%%%%%%%
%%%
%%% Internship
%%%
%%%%%%%%%%%%%%%%%%%%%%%%%%%%%%%%%%%%
%%%%%%%%%%%%%%%%%%%%%%%%%%%%%%%%%%%%


%%%%%%%%%%%%%%%
%%  req form - Internship
%%%%%%%%%%%%%%%
\SetForm{internshiprequest}{heading}{\UseTerm{title.internship}}
\SetForm{internshiprequest}{title}{\UseTerm{title.requestform}}

%%%
\SetForm{internshiprequest}{opening}{

  \UseTerm{opening.requestform}}

%%%
\SetForm{internshiprequest}{body}{
  \UseTerm{desc.requestform} \Activity{internship}{name}.
    \vfill%
   \UseLabel{student.email}: {\bf \DataFields{student}{email}} %
\vspace{2mm}

\UseLabel{tutor}: \textbf{\DataFields{student.tutor}{name}}%

\UseLabel{tutor.email}: {\bf \DataFields{student.tutor}{email}}%
\hfill \UseLabel{phone}: {\bf \DataFields{student.tutor}{phone}}%
\vspace{2mm}


Empresa: {\bf \DataFields{student.internship}{company}}%

Área de Atividade: {\bf \DataFields{student.internship}{field}}%
\vspace{2mm}

\UseTerm{desc.supervisor}

Cargo: {\bf \DataFields{student.supervisor}{office}}\hfill Email: {\bf \DataFields{student.supervisor}{email}} \hfill Fone: {\bf \DataFields{student.supervisor}{phone}}   %
\vspace{2mm}

Data de Início: {\bf \DataFields{student.internship}{start}}\hfill Previsão de Término: {\bf \DataFields{student.internship}{end}}\hfill Duração: {\bf \DataFields{student.internship}{length}}

}

%%%
\SetForm{internshiprequest}{closing}{
  \parbox[t]{.45\textwidth}{\SignatureDate{\UseLabel{signature.supervisor}}}\hfill%
  \parbox[t]{.45\textwidth}{\SignatureDate{\UseLabel{signature.tutor}}}%

    \hfill%
    \parbox[t]{.45\textwidth}{\SignatureDate{\UseLabel{signature.student}}}
}


%%%%%%%%%%%%%%%
%%  supervisor`s form - Internship
%%%%%%%%%%%%%%%
\SetForm{internshipsupervisorform}{heading}{\UseTerm{title.internship}}
\SetForm{internshipsupervisorform}{title}{\UseTerm{title.internsupervisorform}}
\SetForm{internshipsupervisorform}{footnote}{\UseTerm{footnote.supervisorsform}}

%%%
\SetForm{internshipsupervisorform}{opening}{
    \UseTerm{desc.student}\\%
    \noindent\UseTerm{desc.supervisor}\\%
    \UseTerm{desc.internshipperiod}
}

%%%
\SetForm{internshipsupervisorform}{body}{
\begin{itemize}[tcc,nosep]
 \item Grupo I - Aspectos Profissionais
   \begin{enumerate}[tcc,nosep]
     \item \textbf{Qualidade do Trabalho}:\emptybox\\ {\footnotesize considerar a qualidade do trabalho tendo em vista o que seria desejável}
     \item \textbf{Engenhosidade}:\emptybox\\ {\footnotesize capacidade de sugerir, projetar ou executar modificações ou inovações.}
     \item \textbf{Conhecimentos}:\emptybox\\ {\footnotesize conhecimento demonstrado no desenvolvimento das atividades programadas.}
     \item \textbf{Cumprimentos das Tarefas}:\emptybox\\ {\footnotesize     considerar o volume de atividades cumpridas dentro de um padrão razoável.}
     \item \textbf{Iniciativa}:\emptybox\\ {\footnotesize iniciativa para desenvolver suas atividades sem dependências de outros.}
   \end{enumerate}
 \item Grupo II - Aspectos Humanos
    \begin{enumerate}[tcc,nosep]
     \item \textbf{Assiduidade}:\emptybox\\ {\footnotesize cumprimento do horário do estágio e ausência de falhas}
\item \textbf{Disciplina}:\emptybox\\ {\footnotesize observância das normas e regulamentos internos da empresa.}
\item \textbf{Sociabilidade}:\emptybox\\ {\footnotesize facilidade de se integrar com colegas e ambiente de trabalho.}
\item \textbf{Cooperação}:\emptybox\\ {\footnotesize     disposição para cooperar com colegas e para atender a atividades..}
\item \textbf{Senso de Responsabilidade}:\emptybox\\ {\footnotesize zelo pelo material, equipamentos e bens da empresa.}
    \end{enumerate}
\end{itemize}

    \vfill%

    Considerações Complementares: \BlankLines{5}
}

%%%
\SetForm{internshipsupervisorform}{closing}{
\hfill\parbox[t]{.45\textwidth}{\SignatureDate{\UseLabel{signature.supervisor}}}
}

%%%%%%%%%%%%%%%
%%  tutor`s form - Internship
%%%%%%%%%%%%%%%
\SetForm{internshiptutorform}{heading}{\UseTerm{title.internship}}
\SetForm{internshiptutorform}{title}{\UseTerm{title.interntutorform}}

%%%
\SetForm{internshiptutorform}{opening}{
    \UseTerm{desc.student}\\%
    \UseLabel{tutor}: \textbf{\DataFields{student.tutor}{name}}\\%
    \UseTerm{desc.internshipperiod}
}

%%%
\SetForm{internshiptutorform}{body}{
    Avaliação d\DataFields{student.tutor}{article} Prof\DataFields{student.tutor}{narticle}. Tutor\DataFields{student.tutor}{narticle}:\emptybox

    \vfill%

    Considerações d\DataFields{student.tutor}{article} Prof\DataFields{student.tutor}{narticle}. Tutor\DataFields{student.tutor}{narticle} sobre o Estágio: 
    
    \BlankLines{5}%

    \vfill%

    \hfill\parbox[t]{.45\textwidth}{\SignatureDate{\UseLabel{signature.tutor}}}

 %   \hline
    \vfill%

    Avaliação d\ActivityCoord{internship}{article} Prof\ActivityCoord{internship}{narticle}. Coordenador\ActivityCoord{internship}{narticle}:\emptybox%
    \vfill

    Considerações d\ActivityCoord{internship}{article} Prof\ActivityCoord{internship}{narticle}. Coordenador\ActivityCoord{internship}{narticle} sobre o Estágio:
    
     \BlankLines{5}%
    \vfill%

    \hfill\parbox[t]{.45\textwidth}{\SignatureDate{\UseLabel{signature.internshipcoord}}}
}

%%%
\SetForm{internshiptutorform}{closing}{}


%%%%%%%%%%%%%%%%%%%%%%%%%%%%%%%%%%%%
%%%%%%%%%%%%%%%%%%%%%%%%%%%%%%%%%%%%
%%%
%%% TCCs - Calendar - EVENTS
%%%
%%%%%%%%%%%%%%%%%%%%%%%%%%%%%%%%%%%%
%%%%%%%%%%%%%%%%%%%%%%%%%%%%%%%%%%%%
%\SetActivityNewEvent{}


%%%%%%%%%%%%%%%%%%%%%%%%%%%%%%%%%%%%
%%%%%%%%%%%%%%%%%%%%%%%%%%%%%%%%%%%%
%%%
%%% TCCs - Calendar
%%%
%%%%%%%%%%%%%%%%%%%%%%%%%%%%%%%%%%%%
%%%%%%%%%%%%%%%%%%%%%%%%%%%%%%%%%%%%
\SetForm{calendar-I}{heading}{\UseTerm{title.tccI}}
\SetForm{calendar-I}{title}{\UseTerm{title.calendar}}
\SetForm{calendar-I}{opening}{}

%%%
  \ExplSyntaxOn
  \bool_new:N \l__ufrgscca_tmp_bool
\SetForm{calendar-I}{body}{%
  \starray_set_iter_from_hash:nn {activity}{tccI}
  \starray_reset_iter:n {activity.calendar}
  \bool_set_true:N \l__ufrgscca_tmp_bool
  { \footnotesize%
    \begin{center}
%
      \resizebox{\textwidth}{!}{%%
        \begin{tabular}{ccp{11.5cm}}%
        Data/Período & Semana~ Letiva & Evento\\\hline
        \bool_do_while:Nn \l__ufrgscca_tmp_bool
          {
            \textbf{\starray_get_prop:nn {activity.calendar}{date}} & \starray_get_prop:nn {activity.calendar}{week} & \starray_get_prop:nn {activity.calendar}{event}\\\hline
            \starray_next_iter:nF {activity.calendar}{\bool_set_false:N \l__ufrgscca_tmp_bool}
          }
        \end{tabular}}%
%
    \end{center}%
    {\tiny \UseTerm{notes.calendar}}
  }%
}
  \ExplSyntaxOff

%%%
\SetForm{calendar-I}{closing}{\vfill}%
%%%%%%%%%%%%%%%
%%%%%%%%%%%%%%%

\SetForm{calendar-II}{heading}{\UseTerm{title.tccII}}
\SetForm{calendar-II}{title}{\UseTerm{title.calendar}}
\SetForm{calendar-II}{opening}{}

%%%
  \ExplSyntaxOn
%  \bool_new:N \l__ufrgscca_tmp_bool
\SetForm{calendar-II}{body}{%
  \starray_set_iter_from_hash:nn {activity}{tccII}
  \starray_reset_iter:n {activity.calendar}
  \bool_set_true:N \l__ufrgscca_tmp_bool
  { \footnotesize%
    \begin{center}
%
      \resizebox{\textwidth}{!}{%%
        \begin{tabular}{ccp{11.5cm}}%
        Data/Período & Semana~ Letiva & Evento\\\hline
        \bool_do_while:Nn \l__ufrgscca_tmp_bool
          {
            \textbf{\starray_get_prop:nn {activity.calendar}{date}} & \starray_get_prop:nn {activity.calendar}{week} & \starray_get_prop:nn {activity.calendar}{event}\\\hline
            \starray_next_iter:nTF {activity.calendar}{}{\bool_set_false:N \l__ufrgscca_tmp_bool}
          }
        \end{tabular}}%
%
    \end{center}%
    {\tiny \UseTerm{notes.calendar}}
  }%
}
  \ExplSyntaxOff

%%%
\SetForm{calendar-II}{closing}{\vfill}%
%%%%%%%%%%%%%%%
%%%%%%%%%%%%%%%




%%%%%%%%%%%%%%%%%%%%%%%%%%%%%%%%%%%%
%%%%%%%%%%%%%%%%%%%%%%%%%%%%%%%%%%%%
%%%
%%% TCC - I
%%%
%%%%%%%%%%%%%%%%%%%%%%%%%%%%%%%%%%%%
%%%%%%%%%%%%%%%%%%%%%%%%%%%%%%%%%%%%


%%%%%%%%%%%%%%%
%%  req form - I
%%%%%%%%%%%%%%%
\SetForm{request-I}{heading}{\UseTerm{title.tccI}}
\SetForm{request-I}{title}{\UseTerm{title.requestform}}
\SetForm{request-I}{opening}{\UseTerm{opening.requestform}}

%%%
\SetForm{request-I}{body}{%
  \UseTerm{desc.requestform} \Activity{tccI}{name}.
  \vfill%

  \UseLabel{worktitle.temp}: \@title %

  \UseLabel{student.email}: {\bf \DataFields{student}{email}} %

  \UseLabel{advisor}: \DataFields{student.advisor}{name}%

  \UseLabel{advisor.email}: {\bf \DataFields{student.advisor}{email}}%
  \hfill \UseLabel{phone}: {\bf \DataFields{student.advisor}{phone}}%

  {\bf Resumo}:\footnote[1]{\UseTerm{footnote.requestform}}%

  \vfill%

  \DataFields{student}{brief}
}

%%%
\SetForm{request-I}{closing}{%
  \parbox[t]{.45\textwidth}{\SignatureDate{\UseLabel{signature.student}}}\hfill%
  \parbox[t]{.45\textwidth}{\SignatureDate{\UseLabel{signature.advisor}}}%
}%
%%%%%%%%%%%%%%%
%%%%%%%%%%%%%%%


%%%%%%%%%%%%%%%
%%  coadvisor - I
%%%%%%%%%%%%%%%
\SetForm{coadvisor-I}{heading}{\UseTerm{title.tccI}}
\SetForm{coadvisor-I}{title}{\UseTerm{title.coadvisorform}}
\SetForm{coadvisor-I}{opening}{%

  \UseTerm{opening.requestform}
  \coadvisoremptyifnone
}

%%%
\SetForm{coadvisor-I}{body}{%
  \UseTerm{desc.coadvisorform}
  \vfill%

  \UseLabel{worktitle.temp}: \@title %

  \UseLabel{advisor}: \DataFields{student.advisor}{name}

  \UseLabel{coadvisor}: \DataFields{student.coadvisor}{name}%

  \UseLabel{coadvisor.email}: {\bf \DataFields{student.coadvisor}{email}}%
  \hfill \UseLabel{phone}: {\bf \DataFields{student.coadvisor}{phone}}\\%

  {\bf Justificativa}:%
  \vfill

 \DataFields{student.coadvisor}{justification}
 
}

%%%
\SetForm{coadvisor-I}{closing}{%

  \SignatureDate{\UseLabel{signature.student}}

  \parbox[t]{.45\textwidth}{\SignatureDate{\UseLabel{signature.advisor}}}\hfill%
  \parbox[t]{.45\textwidth}{\SignatureDate{\UseLabel{signature.coadvisor}}}%
}%
%%%%%%%%%%%%%%%
%%%%%%%%%%%%%%%


%%%%%%%%%%%%%%%
%%  board approval - I
%%%%%%%%%%%%%%%
\SetForm{boardapproval-I}{heading}{\UseTerm{title.tccI}}
\SetForm{boardapproval-I}{title}{\UseTerm{title.boardapproval}}
\SetForm{boardapproval-I}{opening}{\UseTerm{desc.studentwork.full}}

%%%
\SetForm{boardapproval-I}{body}{  \begin{center}
    \UseLabel{board}
  \end{center}%

%%% TODO: That's do be done... convert it to starray!!!
 
  \BoardDescSignature{Avaliador\DataFields{student.reviewer[2]}{narticle}}{student.reviewer[2]}

  \BoardDescSignature{Avaliador\DataFields{student.reviewer[3]}{narticle}}{student.reviewer[3]}

  \vfill%

  Aprovo a nominata da Banca Examinadora acima relacionada.%
}

%%%
\SetForm{boardapproval-I}{closing}{\hfill\parbox[t]{.45\textwidth}{\SignatureDate{\UseLabel{signature.advisor}}}}
%%%%%%%%%%%%%%%
%%%%%%%%%%%%%%%


%%%%%%%%%%%%%%%
%%  advisor's approval - I
%%%%%%%%%%%%%%%
\SetForm{advisorsapproval-I}{heading}{\UseTerm{title.tccI}}
\SetForm{advisorsapproval-I}{title}{\UseTerm{title.advisorsapproval-I}}
\SetForm{advisorsapproval-I}{opening}{\UseTerm{desc.studentwork.full}}

%%%
\SetForm{advisorsapproval-I}{body}{%
  {\bf Parecer d\DataFields{student.advisor}{article} Orientador\DataFields{student.advisor}{narticle}}\footnote{Preenchimento Obrigatório}

  \vfill%


  \DataFields{student.advisor}{assessment}
}

%%%
\SetForm{advisorsapproval-I}{closing}{
  \SignatureDate{\UseLabel{signature.advisor}}\vfill%
  \SignatureDate{\UseLabel{signature.student}}%

}%
%%%%%%%%%%%%%%%
%%%%%%%%%%%%%%%


%%%%%%%%%%%%%%%
%%  examiner's form - I
%%%%%%%%%%%%%%%
\SetForm{examinersform-I}{heading}{\UseTerm{title.tccI}}
\SetForm{examinersform-I}{title}{\UseTerm{title.reviewersform}}
\SetForm{examinersform-I}{footnote}{\UseTerm{footnote.reviewersform}}
\SetForm{examinersform-I}{opening}{%
  \UseTerm{desc.studentwork}\\[2mm]
  
  \UseLabel{reviewer}: \textbf{\DataFields{student.reviewer}{name}}\\[-4mm]%
  
  {\scriptsize {\textsl{(favor retornar este diretamente à coordenação de TCC.)}}}
  }

%%%
\SetForm{examinersform-I}{body}{%  
   O trabalho submetido\footnotemark[2]:

     \begin{enumerate}[tcc,nosep]
     \item \emptybox\  \textbf{Enquadramento do Tema}. {\footnotesize(peso:2)}\\ {\footnotesize Quão bem o mesmo se enquadra no perfil de um formando em Eng. de Controle e Automação.}
     \item \emptybox\ \textbf{Revisão Bibliográfica}. {\footnotesize(peso:2)}\\ {\footnotesize A revisão apresentada é adequada, suficiente, para o tema do trabalho?}
     \item \emptybox\ \textbf{Abordagem/Plano de Trabalho Proposto}. {\footnotesize(peso:2)}\\ {\footnotesize Considerar se a abordagem, plano de trabalho proposto, é adequada e factivel.}
     \item \emptybox\ \textbf{Redação}. {\footnotesize(peso:1)}\\ {\footnotesize     Considerar a qualidade do trabalho escrito apresentado.}
     \item { O Trabalho Submetido}
     \begin{itemize}[tcc,nosep]
       \item[$\Box$] Está adequado, dispensando correções do texto.
       \item[$\Box$] Está adequado, sendo necessário correções, conforme anotações anexas.
       \item[$\Box$] Necessita ser revisto/re-escrito,\footnotemark[3] e: 
       \begin{itemize}
         \item[$\Box$] basta a revisão final do texto pelo orientador,
         \item[$\Box$] revisarei o trabalho depois de corrigido.
       \end{itemize}
     \end{itemize} 
   \end{enumerate}
   
Outras considerações, parecer (se necessário): \BlankLines{6}

  \footnotetext[2]{Use a escala: \begin{enumerate*}\item[0.]Péssimo  \item[3.]Deficitário \item[6.]Bom \item[8.]Muito Bom \item[10.]Excelente \end{enumerate*}}%
  \footnotetext[3]{Neste caso \DataFields{student}{article} alun\DataFields{student}{article} estará em EXAME e receberá conceito final C ou D, conforme revisão final do texto.}%
}

%%%
\SetForm{examinersform-I}{closing}{%signature.reviewer
  \SignatureDate{\UseLabel{signature.reviewer}}%
}
%%%%%%%%%%%%%%%
%%%%%%%%%%%%%%%


%%%%%%%%%%%%%%%
%%  rectify approval - I
%%%%%%%%%%%%%%%
\SetForm{rectifyapproval-I}{heading}{\UseTerm{title.tccI}}
\SetForm{rectifyapproval-I}{title}{\UseTerm{title.rectifyapproval}}
\SetForm{rectifyapproval-I}{opening}{\UseTerm{desc.studentwork.full}}

%%%
\SetForm{rectifyapproval-I}{body}{  %
  \SetTermKeys{temp.formrefname = {Formulário de Avaliação}}
  \UseTerm{desc.rectifyapproval}

  \vfill%

  \ApprovalSignature[\UseLabel{advisor}]{\DataFields{student.advisor}{name}}

}

%%%
\SetForm{rectifyapproval-I}{closing}{ 

  \SetReviewer{2}
  \ApprovalSignature{\DataFields{student.reviewer}{name}}\vfill

  \SetReviewer{3}
  \ApprovalSignature{\DataFields{student.reviewer}{name}}\vfill

 }
%%%%%%%%%%%%%%%
%%%%%%%%%%%%%%%



%%%%%%%%%%%%%%%%%%%%%%%%%%%%%%%%%%%%
%%%%%%%%%%%%%%%%%%%%%%%%%%%%%%%%%%%%
%%%
%%% TCC - II
%%%
%%%%%%%%%%%%%%%%%%%%%%%%%%%%%%%%%%%%
%%%%%%%%%%%%%%%%%%%%%%%%%%%%%%%%%%%%


%%%%%%%%%%%%%%%
%%  req form - II
%%%%%%%%%%%%%%%
\SetForm{request-II}{heading}{\UseTerm{title.tccII}}
\SetForm{request-II}{title}{\UseTerm{title.requestform}}

\SetForm{request-II}{opening}{

\UseTerm{opening.requestform}}

%%%
\SetForm{request-II}{body}{
  \UseTerm{desc.requestform} \Activity{tccII}{name}.
  \vfill%

  \UseLabel{worktitle.temp}: \@title %

  \UseLabel{student.email}: {\bf \DataFields{student}{email}} %

  \UseLabel{advisor}: \DataFields{student.advisor}{name}%

  \UseLabel{advisor.email}: {\bf \DataFields{student.advisor}{email}}%
  \hfill \UseLabel{phone}: {\bf \DataFields{student.advisor}{phone}}%

  \UseLabel{coadvisor}: \DataFields{student.coadvisor}{name}%

  \UseLabel{coadvisor.email}: {\bf \DataFields{student.coadvisor}{email}}%
  \hfill \UseLabel{phone}: {\bf \DataFields{student.coadvisor}{phone}}%


  \vfill%
  O Tema do Trabalho e Orientador (Co-orientador) seguem sendo os mesmos de TCC I ?
\begin{itemize}[tcc,noitemsep]
 \item[$\Box$] SIM
 \item[$\Box$] NÃO, neste caso \textbf{justifique} a mudança no espaço a seguir:
 \end{itemize}
  

  \DataFields{student}{justification}

}

%%%
\SetForm{request-II}{closing}{
  \parbox[t]{.45\textwidth}{\SignatureDate{\UseLabel{signature.student}}}\hfill%
  \parbox[t]{.45\textwidth}{\SignatureDate{\UseLabel{signature.advisor}}}%
}
%%%%%%%%%%%%%%%
%%%%%%%%%%%%%%%

%%%%%%%%%%%%%%%
%%  coadvisor - II
%%%%%%%%%%%%%%%
\SetForm{coadvisor-II}{heading}{\UseTerm{title.tccII}}
\SetForm{coadvisor-II}{title}{\UseTerm{title.coadvisorform}}
\SetForm{coadvisor-II}{opening}{%

  \UseTerm{opening.requestform}
  \coadvisoremptyifnone
}

%%%
\SetForm{coadvisor-II}{body}{%
  \UseTerm{desc.coadvisorform}
  \vfill%

  \UseLabel{worktitle.temp}: \@title %

  \UseLabel{advisor}: \DataFields{student.advisor}{name}

  \UseLabel{coadvisor}: \DataFields{student.coadvisor}{name}%

  \UseLabel{coadvisor.email}: {\bf \DataFields{student.coadvisor}{email}}%
  \hfill \UseLabel{phone}: {\bf \DataFields{student.coadvisor}{phone}}\\%

  {\bf Justificativa}:%
  \vfill

  \DataFields{student.coadvisor}{justification}
}

%%%
\SetForm{coadvisor-II}{closing}{%

  \SignatureDate{\UseLabel{signature.student}}

  \parbox[t]{.45\textwidth}{\SignatureDate{\UseLabel{signature.advisor}}}\hfill%
  \parbox[t]{.45\textwidth}{\SignatureDate{\UseLabel{signature.coadvisor}}}%
}
%%%%%%%%%%%%%%%
%%%%%%%%%%%%%%%




%%%%%%%%%%%%%%%
%%  board approval - II
%%%%%%%%%%%%%%%
\SetForm{boardapproval-II}{heading}{\UseTerm{title.tccII}}
\SetForm{boardapproval-II}{title}{\UseTerm{title.boardapproval}}
\SetForm{boardapproval-II}{opening}{\UseTerm{desc.studentwork.full}}

%%%
\SetForm{boardapproval-II}{body}{%
  \begin{center}
    \UseLabel{board}%
  \end{center}%

  \BoardDescSignature{Avaliador\DataFields{student.reviewer[1]}{narticle}}{student.reviewer[1]}
  
  \BoardDescSignature{Avaliador\DataFields{student.reviewer[2]}{narticle}}{student.reviewer[2]}
  
  \BoardDescSignature{Avaliador\DataFields{student.reviewer[3]}{narticle}}{student.reviewer[3]}

  \BoardDescSignature{Suplente}{student.altreviewer[1]}

  \vfill%

    Aprovo a nominata da Banca Examinadora acima relacionada.%
}

%%%
\SetForm{boardapproval-II}{closing}{  \SignatureDate{\UseLabel{signature.advisor}}}
%%%%%%%%%%%%%%%
%%%%%%%%%%%%%%%


%%%%%%%%%%%%%%%
%%  advisor's approval - II
%%%%%%%%%%%%%%%
\SetForm{advisorsapproval-II}{heading}{\UseTerm{title.tccII}}
\SetForm{advisorsapproval-II}{title}{\UseTerm{title.advisorsapproval-II}}
\SetForm{advisorsapproval-II}{opening}{\UseTerm{desc.studentwork.full}}

%%%
\SetForm{advisorsapproval-II}{body}{%
  {\bf Parecer d\DataFields{student.advisor}{article} Orientador\DataFields{student.advisor}{narticle}}\footnote{Preenchimento Obrigatório}

  \vfill%

   \DataFields{student.advisor}{assessment}
}

%%%
\SetForm{advisorsapproval-II}{closing}{%
  \SignatureDate{\UseLabel{signature.advisor}}%
  \vfill%

  \SignatureDate{\UseLabel{signature.student}}%
 }
%%%%%%%%%%%%%%%
%%%%%%%%%%%%%%%


%%%%%%%%%%%%%%%
%%  receipts - II
%%%%%%%%%%%%%%%
\SetForm{receipts-II}{heading}{\UseTerm{title.tccII}}
\SetForm{receipts-II}{title}{\UseTerm{title.receipts}}
\SetForm{receipts-II}{footnote}{\UseTerm{footnote.receipts}}

%%%
\SetForm{receipts-II}{opening}{%
  \UseTerm{desc.studentwork.full}\\%

  \noindent Membro da Banca : {\bf \DataFields{student.reviewer}{name} }%
}

%%%
\SetForm{receipts-II}{body}{ %
  Declaro que encontra-se em minha posse cópia da versão final do \Activity{tccII}{name} em \Activity{course}{name}  d\DataFields{student}{article} alun\DataFields{student}{article} 
  \textbf{\DataFields{student}{name}}.

  \vfill%

 \SignatureDate{\UseLabel{signature.board}}%
 
}

%%%
\SetForm{receipts-II}{closing}{}
%%%%%%%%%%%%%%%
%%%%%%%%%%%%%%%


%%%%%%%%%%%%%%%
%%  examiner's form - II
%%%%%%%%%%%%%%%
\SetForm{examinersform-II}{heading}{\UseTerm{title.tccII}}
\SetForm{examinersform-II}{title}{\UseTerm{title.reviewersform}}
\SetForm{examinersform-II}{footnote}{\UseTerm{footnote.reviewersform}}

%%%
\SetForm{examinersform-II}{opening}{\UseTerm{desc.studentwork}
  
  \noindent\UseLabel{reviewer}: \textbf{\DataFields{student.reviewer}{name}}
}

%%%
\SetForm{examinersform-II}{body}{ 
A nota (variando de zero a dez) atribuída por cada membro da banca examinadora será dada considerando o desempenho d\DataFields{student}{article} alun\DataFields{student}{article} nos 3 pontos apresentados na tabela abaixo.

  \vfill%

  \begin{center}
    \begin{tabular}{|c|c|c|}%
      \hline%
      \UseLabel{TCC-II.pointA} & \UseLabel{TCC-II.pointB} & \UseLabel{TCC-II.pointC} \\\hline%
      & & \\\hline
    \end{tabular}%
  \end{center}%
}

%%%
\SetForm{examinersform-II}{closing}{ 
  
 \SignatureDate{\UseLabel{signature.reviewer}}%
  \vfill%
  \begin{description}
    \item[Observações:] \ \\A nota final, de cada membro da banca examinadora, será calculada conforme:\\[2mm] {\bf Nota Final = \UseLabel{TCC-II.weightA}\ x \UseLabel{TCC-II.pointA}\  + \UseLabel{TCC-II.weightB}\ x \UseLabel{TCC-II.pointB}\  + \UseLabel{TCC-II.weightC}\ x \UseLabel{TCC-II.pointC}\ } \\[2mm] A média final da atividade será dada pela média das notas finais atribuidas por cada um dos membros da banca examinadora.
  \end{description}%
}
%%%%%%%%%%%%%%%
%%%%%%%%%%%%%%%


%%%%%%%%%%%%%%%
%%  examiner's form - II
%%%%%%%%%%%%%%%
\SetForm{correctionsform-II}{heading}{\UseTerm{title.tccII}}
\SetForm{correctionsform-II}{title}{\UseTerm{title.correctionsform}}
\SetForm{correctionsform-II}{footnote}{\UseTerm{footnote.correctionsform}}

%%%
\SetForm{correctionsform-II}{opening}{\UseTerm{desc.studentwork}
  
    \noindent\UseLabel{reviewer}: \textbf{\DataFields{student.reviewer}{name}}

}

%%%
\SetForm{correctionsform-II}{body}{  \resizebox{0.9\textwidth}{!}{%%
O texto submetido:
 \parbox[t]{.75\textwidth}{
    \begin{itemize}[tcc,nosep]
     \item[$\Box$] Está adequado, dispensando correções do texto.
     \item[$\Box$] Está adequado, sendo necessário correções, conforme a
      seguir (e anotações anexas).
      	\begin{description}[tcc,noitemsep]
        	\item[$\Box$] Ortográfica
        	\item[$\Box$] Regras de Escrita
        	\item[$\Box$] Equações
        	\item[$\Box$] Resumo
        	\item[$\Box$] Abstract
        	\item[$\Box$] Índice
        	\item[$\Box$] Introdução
        	\item[$\Box$] Conclusões
        	\item[$\Box$] Corpo do Trabalho
        	\item[$\Box$] Resultados
        	\item[$\Box$] Adicionar mais exemplos
        	\item[$\Box$] Figuras
        	\item[$\Box$] Tabelas
        \end{description}
     \item[$\Box$] Necessita ser revisto/re-escrito,\footnotemark[2] e: 
       \begin{itemize}
         \item[$\Box$] basta a revisão final do texto pelo orientador,
         \item[$\Box$] revisarei o trabalho depois de corrigido.
       \end{itemize}
	\item[$\Box$] Outras considerações (se necessário): \\
	\begin{tabular}{p{0.7\textwidth}}
	~\\\hline
	~\\\hline
	~\\\hline
	~\\\hline
	\end{tabular}\\[-1mm]
	\end{itemize}}  }%
  \footnotetext[2]{Neste caso \DataFields{student}{article} alun\DataFields{student}{article} estará em EXAME e receberá conceito final C ou D, conforme revisão final do texto.}%
}

%%%
\SetForm{correctionsform-II}{closing}{  \SignatureDate{\UseLabel{signature.reviewer}}%
 }
%%%%%%%%%%%%%%%
%%%%%%%%%%%%%%%



%%%%%%%%%%%%%%%
%%  rectify approval - II
%%%%%%%%%%%%%%%
\SetForm{rectifyapproval-II}{heading}{\UseTerm{title.tccII}}
\SetForm{rectifyapproval-II}{title}{\UseTerm{title.rectifyapproval}}
\SetForm{rectifyapproval-II}{opening}{\UseTerm{desc.studentwork.full}}

%%%
\SetForm{rectifyapproval-II}{body}{  %
  \SetTermKeys{temp.formrefname = {Formulário de Correções}}
  \UseTerm{desc.rectifyapproval}

  \vfill%

  \ApprovalSignature[\UseLabel{advisor}]{\DataFields{student.advisor}{name}}
}

%%%
\SetForm{rectifyapproval-II}{closing}{%
 \@ifundefined{tc@nBoard}{}{%
  \SetReviewer{1}
  \ApprovalSignature{\DataFields{student.reviewer}{name}}\vfill
  }%

  \SetReviewer{2}
  \ApprovalSignature{\DataFields{student.reviewer}{name}}\vfill

  \SetReviewer{3}
  \ApprovalSignature{\DataFields{student.reviewer}{name}}\vfill


  \ApprovalSignature[\UseLabel{altreviewer.sig}]{\DataFields{student.altreviewer}{name}}
}
%%%%%%%%%%%%%%%
%%%%%%%%%%%%%%%
%%
%%%%%%%%%%%%%%%
% examiners approval (or not) signature (for rectify approval form)%
%%%%%%%%%%%%%%%
\ExplSyntaxOn
\NewDocumentCommand{\ApprovalSignature}{O{\UseLabel{reviewer}}m}{%%
  \UseTerm{corrections.ok}  \hfill Data:\rule{\l__ufrgscca_date_rule_dim}{.1pt}\\[3mm]%
  #1: #2 \hfill  \UseLabel{signature.gen}:\rule{\l__ufrgscca_name_rule_dim}{.1pt}%
}%
\ExplSyntaxOff



