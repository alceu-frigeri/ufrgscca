%%%==============================================================================
%% Copyright 2022-present by Alceu Frigeri
%%
%% This work may be distributed and/or modified under the conditions of
%%
%% * The [LaTeX Project Public License](http://www.latex-project.org/lppl.txt),
%%   version 1.3c (or later), and/or
%% * The [GNU Affero General Public License](https://www.gnu.org/licenses/agpl-3.0.html),
%%   version 3 (or later)
%%
%% This work has the LPPL maintenance status *maintained*.
%%
%% The Current Maintainer of this work is Alceu Frigeri
%%
%% This is version {2.4} {2024/02/19}
%%
%% The list of files that compose this work can be found in the README.md file at
%% https://ctan.org/pkg/ufrgscca
%%
%%%==============================================================================
%% UFRGS stands for "Federal University of Rio Grande do Sul" in south Brazil
%% EE    stands for "Engineering School"
%% CCA   stands for "Control and Automation Engineering Course" (Portuguese acronym)
%%%==============================================================================


\location{Porto~Alegre}{RS}

\SetHeadings{%
    university              = Universidade Federal do Rio Grande do Sul ,
    acronym                 = UFRGS ,
    unit                    = Escola de Engenharia ,
    secretary               = SACAD/EE ,
    course                  = Engenharia de Controle e Automação ,
    course.title            = Bacharel em Engenharia de Controle e Automação ,
    ppc.comgrad             = {COMISSÃO DE GRADUAÇÃO - COMGRAD/CCA} ,
    ppc.nde                 = {NÚCLEO DOCENTE ESTRUTURANTE - NDE/CCA} ,
    ppc.course              = {DO CURSO EM ENGENHARIA DE CONTROLE E AUTOMAÇÃO} ,
    department              = {Depto. de Sistemas Elétricos de Automação e Energia} ,
}



\ActivitySelect{course}
\ActivitySet{CCA}{Eng. de Controle e Automação}
\ActivitySetCoord{Nome Coordenador}
\ActivitySetCoordTitle{Coordenador\ActivityCoord[course]{narticle} de Curso}

\ActivitySelect{tccI}
\ActivitySet{TCC~-~I}{Trabalho de Conclusão de Curso CCA - I}
\ActivitySetCoord{Nome Coordenador}
\ActivitySetCoordTitle{Coordenador\ActivityCoord[tccI]{narticle} da Atividade de Ensino de\break Trabalho de Conclusão de Curso}

\ActivitySelect{tccII}
\ActivitySet{TCC~-~II}{Trabalho de Conclusão de Curso CCA - II}
\ActivitySetCoord{Nome Coordenador}
\ActivitySetCoordTitle{Coordenador\ActivityCoord[tccII]{narticle} da Atividade de Ensino de\break Trabalho de Conclusão de Curso}

\ActivitySelect{internship}
\ActivitySet{Est.Ob.}{Estágio Supervisionado Obrigatório}
\ActivitySetCoord{Nome Coordenador}
\ActivitySetCoordTitle{Coordenador\ActivityCoord[internship]{narticle} da Atividade de Ensino de\newline Estágio Supervisionado Obrigatório}

\ActivitySelect{internship-opt}
\ActivitySet{Est. Não Ob.}{Estágio Supervisionado Não Obrigatório}
\ActivitySetCoord{Nome Coordenador}
\ActivitySetCoordTitle{Coordenador\ActivityCoord[internship]{narticle} da Atividade de Ensino de\newline Estágio Supervisionado Obrigatório}


\SetTerms{%
  title.tcc                 = Trabalho de Conclusão de Curso ,
  title.tccI                = Trabalho de Conclusão de Curso I ,
  title.tccII               = Trabalho de Conclusão de Curso II ,  
  title.internship          = Estágio Supervisionado Obrigatório ,
  %
  title.calendar            = {Calendário - \UseTerm{semester}} , 
  title.requestform         = Formulário de Requerimento de Matrícula ,
  title.coadvisorform       = Requerimento de Autorização de Coorientação ,
  title.boardapproval       = Formulário de Indicação da Banca Examinadora ,
  title.advisorsapproval-I  = Formulário de Aprovação para Avaliação ,
  title.advisorsapproval-II = Formulário de Aprovação para Apresentação ,
  title.receipts            = Recibo de entrega de cópia da versão final do Trabalho de Conclusão ,
  title.reviewersform       = Formulário de Avaliação ,
  title.correctionsform     = Formulário de Correções ,
  title.rectifyapproval     = Formulário de Aprovação das Correções ,
  title.internsupervisorform = Formulário de Avaliação do Supervisor (Pela Empresa) ,
  title.interntutorform      = Formulário de Avaliação do Tutor e Coordenador ,
  %
  title.attendancelist       = {Lista de Preseça - 1$^o$ Encontro do Semestre \UseTerm{semester}} ,
  %
  %footnote.reviewersform
  footnote.reviewersform    = {Entregar a cada membro da banca juntamente com a cópia do trabalho.} , %%% TCC I / II
  footnote.receipts         = {Entregar a cada membro da banca juntamente com a cópia do trabalho final e após encaminhar à coordenação de TCC.} , %%% TCC II
  footnote.correctionsform  = {Entregar a cada membro da banca juntamente com a cópia do trabalho. Após apresentação, retornar este à coordenação de TCC.} ,%%% TCC II
   %
  footnote.supervisorsform = {Favor utilizar a seguinte grade discreta: \begin{enumerate*}[start=0]\item Sofrível \item Regular \item Bom \item Muito Bom \item Excelente \end{enumerate*}} ,
   %
  opening.requestform       = { Ilm\ActivityCoord[course]{article}. Sr\ActivityCoord[course]{narticle}.\par
 Coordenador\ActivityCoord[course]{narticle} da  Comissão de Graduação do Curso em \Activity[course]{name}\  da \UseHeading{university}.   } ,
  localdate                 = {\UseConst{publ.city},\hspace{4mm} de \hspace{1.2cm} de 20\hspace{4mm}.} ,
  %
  %
  %
  %
 footnote.requestform       = { O resumo deve ser curto, i.e., não deve exceder este espaço. O texto deve ser escrito em parágrafo único, 
                                e por se tratar de proposta, poderá utilizar  tempo futuro e não necessita ter motivação ou justificativa.\break  Deve conter:
                                \begin{enumerate*}[alpha)] 
                                	\item O QUE será feito - descrição e objetivos.
                                   \item COMO será feito - metodologia.
                                	\item Resultados esperados.
                                \end{enumerate*}%
                                } ,   %%% TCC I
  desc.requestform          = { \DataFields{student}{Article} alun\DataFields{student}{article} \DataFields{student}{name}, 
                              REGULARMENTE MATRICULAD\DataFields{student}{Article} neste período letivo no curso \emph{\Activity[course]{name}}, desta universidade, 
                              portador\DataFields{student}{narticle} do cartão de identificação n$^o$ \DataFields{student}{ID}, 
                              vem por meio deste solicitar matrícula na atividade de ensino} ,
  desc.coadvisorform            = {\DataFields{student}{Article} alun\DataFields{student}{article} \DataFields{student}{name}, 
                              REGULARMENTE MATRICULAD\DataFields{student}{Article} na atividade de TCC neste período letivo, 
                              vem por meio deste solicitar autorização para registro de coorientação de TCC, conforme justificativa abaixo.} ,
  desc.rectifyapproval      = { \begin{description}
                              \item[Observação 1:] Caso a versão final corrigida do trabalho, mais este formulário, não sejam entregues até a data estipulada, o conceito final passará a ser {\bf D}.
                              \item[Observação 2:] Caso as correções sejam satisfatórias, a nota final será a já atribuída pela Banca Examinadora. 
                              \item[Observação 3:] A assinatura d\DataFields{student.advisor}{article} orientador\DataFields{student.advisor}{narticle} é obrigatória, entretanto a assinatura dos membros da banca só é necessária caso tenham marcado a opção ''$\Box$ Revisarei o trabalho depois de corrigido'' no \emph{\UseTerm{temp.formrefname}}.
                            \end{description}
                              } ,
  desc.studentwork          = { 
                              \noindent\UseLabel{student}: {\bf \DataFields{student}{name}}\hfill \UseLabel{student.ID}: {\bf \DataFields{student}{ID}}\\%
                              \noindent\UseLabel{worktitle}: \DataFields{student}{worktitle} } , 
  desc.studentwork.full     = {\UseTerm{desc.studentwork}\\[2.5mm]\UseTerm{desc.advisor}} ,  
  desc.advisor              = {\noindent\UseLabel{advisor}:  \textbf{\DataFields{student.advisor}{name}}} ,
  desc.coadvisor            = {\noindent\UseLabel{coadvisor}:  \textbf{\DataFields{student.advisor}{name}}} ,
  desc.reviewer             = {\noindent\UseLabel{reviewer}:  \textbf{\DataFields{student.advisor}{name}}} ,
  desc.student              = {\noindent\UseLabel{student}: {\bf \DataFields{student}{name}}\hfill 
                              \UseLabel{student.ID}: {\bf \DataFields{student}{ID}}} ,
  desc.internshipperiod     = {\noindent Período do Estágio: {\bf \DataFields{student.internship}{start}} até: {\bf \DataFields{student.internship}{end}}} ,
  desc.supervisor           = {\UseLabel{supervisor}: \textbf{\DataFields{student.supervisor}{name}} \hfill CREA: \textbf{\DataFields{student.supervisor}{register}}} ,
  desc.sacad                = {\noindent Ao\par\noindent SACAD da Escola de Engenharia} ,
  corrections.ok            = {Correções satisfatórias: \hspace*{1cm}$\Box$\ sim\hspace*{2cm}$\Box$\ não} , 
}



\SetTerms{%
  tccI.text-descpage         = 
      {  Proposta de Trabalho de Conclusão de Curso submetida à COMGRAD/\Activity[course]{acronym}\   da \UseHeading{acronym}\  
         como parte dos requisitos para a obtenção do título de \emph{\UseHeading{course.title}.}%
      } ,
  tccI.text-approvalpage     = 
      { Esta Proposta de Trabalho de Conclusão de Curso foi julgada adequada para a obtenção dos créditos da Atividade 
        de \emph{\Activity[tccI]{name}} e aprovada em sua forma final \studentCoadvCase{pelos Orientadores}{pel\DataFields{student.advisor}{article} Orientador\DataFields{student.advisor}{narticle}} e Banca Examinadora abaixo.%
      } ,                               
  tccII.text-descpage         = 
      { Trabalho de Conclusão de Curso submetido à COMGRAD/\Activity[course]{acronym}\   da \UseHeading{acronym}\  como 
        parte dos requisitos para a obtenção do título de \emph{\UseHeading{course.title}.}%
      } ,
  tccII.text-approvalpage    = 
      { Este Trabalho de Conclusão de Curso foi julgado adequado para a obtenção dos créditos da Atividade 
        de \emph{\Activity[tccII]{name}} e aprovado em sua forma final \studentCoadvCase{pelos Orientadores}{pel\DataFields{student.advisor}{article}
        Orientador\DataFields{student.advisor}{narticle}} e Banca Examinadora abaixo.%
      } ,
  internship.title = 
      {   \centering\textbf{%
          RELATÓRIO DE ESTÁGIO SUPERVISIONADO OBRIGATÓRIO, DESENVOLVIDO\\DE \DataFields{student.internship}{start} ATÉ \DataFields{student.internship}{end} }  
      },
  internship.text-approvalpage = 
      { Este Relatório foi julgado adequado para a obtenção do conceito final na Atividade de Estágio Supervisionado Obrigatório em 
        Eng. de Controle e Automação e aprovado em sua forma final pel\DataFields{student.tutor}{article} 
        Tutor\DataFields{student.tutor}{narticle}, Supervisor\DataFields{student.supervisor}{narticle} e 
        Coordenador\DataFields{activity[internship].coord}{narticle} da Atividade.%
      } , 
  internship-opt.text-approvalpage = 
      { Este Relatório foi julgado adequado para uma Atividade de Estágio Supervisionado Não Obrigatório em Eng. de Controle
        e Automação e aprovado em sua forma final pel\DataFields{student.tutor}{article} Tutor\DataFields{student.tutor}{narticle} e Supervisor\DataFields{student.supervisor}{narticle}.%
      } ,
  internship-opt.title = 
      {   \centering\textbf{%
          RELATÓRIO DE ESTÁGIO SUPERVISIONADO NÂO OBRIGATÓRIO, DESENVOLVIDO\\DE \DataFields{student.internship}{start} ATÉ \DataFields{student.internship}{end} }  
      },
}




%\SetTerms{%
%  
%}


\SetLabels{%
  professor                 = {Professor\DataFields{student.advisor}{narticle}} ,
  professors                = {Professores} ,
  advisor                   = {Orientador\DataFields{student.advisor}{narticle}} ,
  advisors                  = {Orientadores} ,
  advisor.gen               = {Orientador(a)} ,
  advisor.email             = {Email d\DataFields{student.advisor}{article} Orientador\DataFields{student.advisor}{narticle}} ,
  coadvisor                 = {Coorientador\DataFields{student.coadvisor}{narticle}} ,
  coadvisor.gen             = {Coorientador(a)} ,
  coadvisor.email           = {Email d\DataFields{student.coadvisor}{article} Coorientador\DataFields{student.coadvisor}{narticle}} ,
  reviewer.base             = {Avaliador} ,
  reviewer                  = {Avaliador\DataFields{student.reviewer}{narticle}} ,
  altreviewer               = {Suplente} ,  
  altreviewer.sig           = {Avaliador\DataFields{student.altreviewer}{narticle}} ,
  tutor                     = {Tutor\DataFields{student.tutor}{narticle}} ,
  tutor.gen                 = {Tutor(a)} ,
  tutor.email               = {Email d\DataFields{student.tutor}{article} Tutor\DataFields{student.tutor}{narticle}} ,  
  supervisor                = {Supervisor\DataFields{student.supervisor}{narticle}} ,
  supervisor.gen            = {Supervisor(a)} ,  
  student                   = {Alun\DataFields{student}{article}} ,
  student.gen               = {Aluno(a)} ,
  student.email             = {Email d\DataFields{student}{article} Alun\DataFields{student}{article}} ,
  student.ID                = {Cartão} ,
  signature.gen             = {Assinatura} , 
  signature.student         = {Assinatura\ \ d\DataFields{student}{article}\ \ Alun\DataFields{student}{article}} ,
  signature.student.gen     = {Assinatura\ \ do(a)\ \ Aluno(a)} ,
  signature.advisor         = {Assinatura\ \ d\DataFields{student.advisor}{article}\ \ Orientador\DataFields{student.advisor}{narticle}} ,
  signature.coadvisor       = {Assinatura\ \ d\DataFields{student.coadvisor}{article}\ \ Coorientador\DataFields{student.coadvisor}{narticle}} ,
  signature.reviewer        = {Assinatura\ \ d\DataFields{student.reviewer}{article}\ \ Avaliador\DataFields{student.reviewer}{narticle}} ,
  signature.board           = {Assinatura\ \ do\ \ Membro\ \ da\ \ Banca} ,
  signature.gen             = {Assinatura} ,
  signature.supervisor      = {Assinatura\ \ d\DataFields{student.supervisor}{article}\ \ Supervisor\DataFields{student.supervisor}{narticle}} ,
  signature.tutor           = {Assinatura\ \ d\DataFields{student.tutor}{article}\ \ Tutor\DataFields{student.tutor}{narticle}} ,
  signature.internshipcoord = {Assinatura\ \ d\ActivityCoord[internship]{article}\ \ Coordenador\ActivityCoord[internship]{narticle}} ,
  worktitle                 = {Título do Trabalho} ,
  worktitle.temp            = {Título Provisório} ,
  email                     = {Email} ,
  phone                     = {ramal} ,
  board                     = {Banca Examinadora} ,
  board.member              = {Membro da Banca} ,
  agree                     = {De Acordo} ,
  Nproc                     = {SEI} ,
  remarks                   = {remarks} ,
  boardtime.gen             = {Horário} , 
}

\SetLabels{%
  TCC-I.pointA               = {Tema} ,
  TCC-I.pointB               = {Biblio.} ,
  TCC-I.pointC               = {Plano Trab.} ,
  TCC-I.pointD               = {Texto} ,
  TCC-II.pointA              = {Conteúdo} , 
  TCC-II.pointB              = {Trab. Escrito} ,
  TCC-II.pointC              = {Apresentação} ,
  TCC-II.pointD              = {} ,
  Ngrade                     = {Nota} ,
  grade                      = {Conceito} ,
}

\SetConsts{%
  TCC-I.weightA              = {2} ,
  TCC-I.weightB              = {2} ,
  TCC-I.weightC              = {2} ,
  TCC-I.weightD              = {1} ,
  TCC-II.weightA             = {0.35} , 
  TCC-II.weightB             = {0.35} ,
  TCC-II.weightC             = {0.3} ,
  NgradeA                    = {9.0} ,
  NgradeB                    = {7.5} ,
  NgradeC                    = {6.0} ,  
}



\SetConsts{%
  report.tabA = {40mm} ,
  report.tabA* = {44mm} ,
  report.tabB = {42mm} ,
  report.tabB* = {62mm} ,
  report.tabC = {15mm} ,
  report.tabD = {15mm} ,
  report.tabE = {15mm} ,
  report.tabEx = {15mm} ,
  report.tabF = {8mm} ,
  report.tabG = {12mm} , 
  report.tabMult.II = {141mm} , % A+B+C+D+E+F+G
  report.tabMult.I = {156mm} , % A+B+C+D+E+F+G + Ex
}

\SetConsts{%
  board.tabA = {38mm} ,
  board.tabB = {38mm} ,
  board.tabC = {60mm} ,
  board.tabD = {13mm} ,
  board.tabAll = {148mm} ,
}

\SetConsts{%
  cover.winwidth          = {95mm} ,
  cover.winheight         = {64mm} ,
  cover.winvoffset        = {112mm} ,
  form.winvoffset         = {24mm} ,
}

